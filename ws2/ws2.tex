	
% This template from http://www.vel.co.nz, originally from http://www.tedpavlic.com

\documentclass{article}
% Change "article" to "report" to get rid of page number on title page
\usepackage{amsmath,amsfonts,amsthm,amssymb, mathrsfs}
\usepackage{bigints}
\usepackage{setspace}
\usepackage{Tabbing}
\usepackage{fancyhdr}
\usepackage{lastpage}
\usepackage{textcomp}
\usepackage{extramarks}
\usepackage{chngpage}
\usepackage{soul,color}
\usepackage{graphicx,float,wrapfig}
\usepackage{cancel}
\usepackage{indentfirst}
\usepackage{mdframed}

% In case you need to adjust margins:
\topmargin=-0.45in      %
\evensidemargin=0in     %
\oddsidemargin=0in      %
\textwidth=6.5in        %
\textheight=9.0in       %
\headsep=0.25in         %

% Homework Specific Information
\newcommand{\hmwkTitle}{WS2}
\newcommand{\hmwkDueDate}{}
\newcommand{\hmwkClass}{Ay\ 190}
\newcommand{\hmwkAuthorName}{Cutter\ Coryell}

% Setup the header and footer
\pagestyle{fancy}                                                       %
\lhead{\hmwkAuthorName}                                                 %
\chead{\hmwkClass\ : \hmwkTitle}  %
\rhead{\hmwkDueDate}                                                     %
\renewcommand\headrulewidth{0.4pt}                                      %
\renewcommand\footrulewidth{0.4pt}                                      %

% This is used to trace down (pin point) problems
% in latexing a document:
%\tracingall

%%%%%%%%%%%%%%%%%%%%%%%%%%%%%%%%%%%%%%%%%%%%%%%%%%%%%%%%%%%%%
% Some tools
\newcommand{\enterProblemHeader}[1]{\nobreak\extramarks{#1}{#1 continued on next page\ldots}\nobreak%
                                    \nobreak\extramarks{#1 (continued)}{#1 continued on next page\ldots}\nobreak}%
\newcommand{\exitProblemHeader}[1]{\nobreak\extramarks{#1 (continued)}{#1 continued on next page\ldots}\nobreak%
                                   \nobreak\extramarks{#1}{}\nobreak}%

\newlength{\labelLength}
\newcommand{\labelAnswer}[2]
  {\settowidth{\labelLength}{#1}%
   \addtolength{\labelLength}{0.25in}%
   \changetext{}{-\labelLength}{}{}{}%
   \noindent\fbox{\begin{minipage}[c]{\columnwidth}#2\end{minipage}}%
   \marginpar{\fbox{#1}}%

   % We put the blank space above in order to make sure this
   % \marginpar gets correctly placed.
   \changetext{}{+\labelLength}{}{}{}}%

\setcounter{secnumdepth}{0}
\newcommand{\homeworkProblemName}{}%
\newcounter{homeworkProblemCounter}%
\newenvironment{homeworkProblem}[1][Problem \arabic{homeworkProblemCounter}]%
  {\stepcounter{homeworkProblemCounter}%
   \renewcommand{\homeworkProblemName}{#1}%
   \section{\homeworkProblemName}%
   \enterProblemHeader{\homeworkProblemName}}%
  {\exitProblemHeader{\homeworkProblemName}}%

\newcommand{\problemAnswer}[1]
  {\noindent\fbox{\begin{minipage}[c]{\columnwidth}#1\end{minipage}}}%

\newcommand{\problemLAnswer}[1]
  {\labelAnswer{\homeworkProblemName}{#1}}

\newcommand{\homeworkSectionName}{}%
\newlength{\homeworkSectionLabelLength}{}%
\newenvironment{homeworkSection}[1]%
  {% We put this space here to make sure we're not connected to the above.
   % Otherwise the changetext can do funny things to the other margin

   \renewcommand{\homeworkSectionName}{#1}%
   \settowidth{\homeworkSectionLabelLength}{\homeworkSectionName}%
   \addtolength{\homeworkSectionLabelLength}{0.25in}%
   \changetext{}{-\homeworkSectionLabelLength}{}{}{}%
   \subsection{\homeworkSectionName}%
   \enterProblemHeader{\homeworkProblemName\ [\homeworkSectionName]}}%
  {\enterProblemHeader{\homeworkProblemName}%

   % We put the blank space above in order to make sure this margin
   % change doesn't happen too soon (otherwise \sectionAnswer's can
   % get ugly about their \marginpar placement.
   \changetext{}{+\homeworkSectionLabelLength}{}{}{}}%

\newcommand{\sectionAnswer}[1]
  {% We put this space here to make sure we're disconnected from the previous
   % passage

   \noindent\fbox{\begin{minipage}[c]{\columnwidth}#1\end{minipage}}%
   \enterProblemHeader{\homeworkProblemName}\exitProblemHeader{\homeworkProblemName}%
   \marginpar{\fbox{\homeworkSectionName}}%

   % We put the blank space above in order to make sure this
   % \marginpar gets correctly placed.
   }%

\newenvironment{myindentpar}[1]%
 {\begin{list}{}%
         {\setlength{\leftmargin}{#1}}%
         \item[]%
 }
 {\end{list}}

%%%%%%%%%%%%%%%%%%%%%%%%%%%%%%%%%%%%%%%%%%%%%%%%%%%%%%%%%%%%%


%%%%%%%%%%%%%%%%%%%%%%%%%%%%%%%%%%%%%%%%%%%%%%%%%%%%%%%%%%%%%
% Make title
\title{\vspace{2in}\textmd{\textbf{\hmwkClass:\ \hmwkTitle}}\\\normalsize\vspace{0.1in}\small{Due\ on\ \hmwkDueDate}\\\vspace{0.1in}\large{\textit{\hmwkClassInstructor\ \hmwkClassTime}}\vspace{3in}}
\date{}
\author{\textbf{\hmwkAuthorName}}
%%%%%%%%%%%%%%%%%%%%%%%%%%%%%%%%%%%%%%%%%%%%%%%%%%%%%%%%%%%%%

%%%% MY COMMANDS %%%%%%%%%%%%%%%%%%%%%

\newcommand{\deri}[2]{\frac{\mathrm{d} #1}{\mathrm{d} #2}}
\newcommand{\pderi}[2]{\frac{\partial #1}{\partial #2}}
\newcommand{\inte}[4]{\int_{#1}^{#2} \! #3 \, \mathrm{d} #4}
\newcommand{\ointe}[4]{\oint_{#1}^{#2} \! #3 \, \mathrm{d} #4}
\newcommand{\del}{\nabla}
\newcommand{\D}{\mathrm{d}}
\newcommand{\ee}[1]{\times 10^{#1}}
\newcommand{\fpe}{\frac{1}{4 \pi \epsilon_0}}
\newcommand{\bra}[1]{\left< #1 \right|}
\newcommand{\ket}[1]{\left| #1 \right>}
\newcommand{\cket}[1]{\left. #1 \right>}


% Distance units
\newcommand{\m}[0]{\text{\ m}}
\newcommand{\cm}[0]{\text{\ cm}}
\newcommand{\km}[0]{\text{\ km}}
\newcommand{\pc}[0]{\text{\ pc}}
\newcommand{\kpc}[0]{\text{\ kpc}}
\newcommand{\Mpc}[0]{\text{\ Mpc}}
\newcommand{\Gpc}[0]{\text{\ Gpc}}
\newcommand{\lyr}[0]{\text{\ lyr}}
\newcommand{\Rs}[0]{R_\odot}

% Mass units
\newcommand{\g}[0]{\text{\ g}}
\newcommand{\kg}[0]{\text{\ kg}}
\newcommand{\Ms}[0]{M_\odot}

% Time units
\newcommand{\s}[0]{\text{\ s}}
\newcommand{\days}[0]{\text{\ days}}
\newcommand{\yr}[0]{\text{\ yr}}
\newcommand{\Hz}[0]{\text{\ Hz}}
\newcommand{\kHz}[0]{\text{\ kHz}}
\newcommand{\MHz}[0]{\text{\ MHz}}
\newcommand{\GHz}[0]{\text{\ GHz}}
\newcommand{\THz}[0]{\text{\ THz}}

% Energy units
\newcommand{\erg}[0]{\text{\ erg}}
\newcommand{\J}[0]{\text{\ J}}
\newcommand{\eV}[0]{\text{\ eV}}
\newcommand{\meV}[0]{\text{\ meV}}
\newcommand{\keV}[0]{\text{\ keV}}
\newcommand{\MeV}[0]{\text{\ MeV}}
\newcommand{\GeV}[0]{\text{\ GeV}}
\newcommand{\TeV}[0]{\text{\ TeV}}

% Force units
\newcommand{\N}[0]{\text{\ N}}
\newcommand{\dyn}[0]{\text{\ dyn}}

% Power units
\newcommand{\W}[0]{\text{\ W}}
\newcommand{\Ls}[0]{L_\odot}

% Temperature units
\newcommand{\K}[0]{\text{\ K}}
\newcommand{\degC}[0]{\text{\ \(^\circ\)C}}
\newcommand{\degF}[0]{\text{\ \(^\circ\)F}}

% Electromagnetic units
\newcommand{\V}[0]{\text{\ V}}
\newcommand{\kV}[0]{\text{\ kV}}
\newcommand{\C}[0]{\text{\ C}}
\newcommand{\esu}[0]{\text{\ esu}}
\newcommand{\T}[0]{\text{\ T}}
\newcommand{\G}[0]{\text{\ G}}


\newcount\colveccount
\newcommand*\colvec[1]{
        \global\colveccount#1
        \begin{pmatrix}
        \colvecnext
}
\def\colvecnext#1{
        #1
        \global\advance\colveccount-1
        \ifnum\colveccount>0
                \\
                \expandafter\colvecnext
        \else
                \end{pmatrix}
        \fi
}

%%%%%%%%%%%%%%%%%%%%%%%%%%%%%%%%%%

\begin{document}
\begin{spacing}{1.1}

\newpage

% When problems are long, it may be desirable to put a \newpage or a
% \clearpage before each homeworkProblem environment

\subsection{1. An Unstable Calculation}

\begin{table}[H]
	\label{table1}
	\centering
    \begin{tabular}{|l|l|l|l|l|}
	\hline
	
	\(n\) & Closed-Form Value & Recursive Value & Absolute Error & Relative Error \\

	\hline

    0 & 1.0 & 1.0 & 0.0 & 0.0 \\
    1 & 0.333333333333 & 0.333333 & 9.93410748107e-09 & 2.98023224432e-08 \\
    2 & 0.111111111111 & 0.111111 & 5.29819064732e-08 & 4.76837158259e-07 \\
    3 & 0.037037037037 & 0.0370373 & 2.16342784749e-07 & 5.84125518823e-06 \\
    4 & 0.0123456790123 & 0.0123466 & 8.71809912319e-07 & 7.06166028978e-05 \\
    5 & 0.00411522633745 & 0.00411871 & 3.48814414362e-06 & 0.0008476190269 \\
    6 & 0.00137174211248 & 0.00138569 & 1.39522571909e-05 & 0.0101711954921 \\
    7 & 0.000457247370828 & 0.000513056 & 5.58089223023e-05 & 0.122054113075 \\
    8 & 0.000152415790276 & 0.000375651 & 0.000223235614917 & 1.46464886947 \\
    9 & 5.08052634253e-05 & 0.000943748 & 0.000892942551319 & 17.5757882376 \\
    10 & 1.69350878084e-05 & 0.00358871 & 0.00357177023583 & 210.909460655 \\
    11 & 5.64502926948e-06 & 0.0142927 & 0.0142870812639 & 2530.91358466 \\
    12 & 1.88167642316e-06 & 0.0571502 & 0.0571483257835 & 30370.9634027 \\
    13 & 6.27225474386e-07 & 0.228594 & 0.228593318278 & 364451.584977 \\
    14 & 2.09075158129e-07 & 0.914374 & 0.914373307961 & 4373419.18641 \\
    \bf{15} & \bf{6.96917193763e-08} & \bf{3.65749} & \bf{3.6574932832} & \bf{52481030.9737} \\

	\hline
	\end{tabular}
	\caption{Closed-form values and recursive values for the expression \(\left(\frac{1}{3}\right)^n\) as well as errors of the recursive value compared to the closed-form value.}
	\end{table}
\newpage
\subsection{2. Finite Differnce Approximation and Convergence}

\begin{figure}[H]
 \label{fig2-1}
 \includegraphics[width=\textwidth]{problem2_fig1.pdf}
 \caption{A plot of \(f'(x;h) - f'(x)\), or the difference between the finite difference of \(f(x) = x^3 - 5x^2 + x\) and the derivative of that function, on the interval \([-2,6]\) for both forward differencing and central differencing, each at two values of the step-size \(h\).}
\end{figure} 

\begin{figure}[H]
 \label{fig2-2}
 \includegraphics[width=\textwidth]{problem2_fig2.pdf}
 \caption{A plot of the ratios of the differencing errors between \(h_1 = 1\) and \(h_2 = 2\), for both forward and central differencing. We expect these ratios to be equal to the convergence factor \(\left( \frac{h_2}{h_1} \right)^n = 2^n\), where \(n\) is the convergence order of the finite differencing: \(n = 1\) for forward differencing and \(n = 2\) for central differencing. Indeed, we see that the ratio is approximately 2 for forward differencing and 4 for central differencing as expected.}
\end{figure} 

\newpage

\subsection{3. Second Derivative}

\[
f(x+h) = f(x) + h f'(x) + \frac{h^2}{2} f''(x) + \frac{h^3}{6} f'''(x) + \mathcal{O}(h^4)
\]
\[
f(x-h) = f(x) - h f'(x) + \frac{h^2}{2} f''(x) - \frac{h^3}{6} f'''(x) + \mathcal{O}(h^4)
\]
\[
\Rightarrow f(x+h) + f(x-h) = 2f(x) + h^2 f''(x) + \mathcal{O}(h^4)
\]
\begin{equation}
\boxed{f''(x) = \frac{f(x+h) - 2f(x) + f(x-h)}{h^2} + \mathcal{O}(h^2)}
\end{equation}

\vspace{1cm}

\subsection{4. Interpolation: Cepheid Lightcurve}

\begin{figure}[H]
 \label{fig4-1}
 \includegraphics[width=\textwidth]{problem4_fig1.pdf}
 \caption{Apparent magnitude versus time for a Cepheid star. Actual measurements are represented as black points. Drawn in red is a Lagrange interpolation through the data.}
\end{figure} 

\begin{figure}[H]
 \label{fig4-2}
 \includegraphics[width=\textwidth]{problem4_fig2.pdf}
 \caption{Apparent magnitude versus time for a Cepheid star. Actual measurements are represented as black points. Drawn in red is a piecewise-linear interpolation through the data, and drawn in blue is a piecewise-quadratic interpolation through the data.}
\end{figure} 

\subsection{5. More Cepheid Lightcurve Interpolation}

\begin{figure}[H]
 \label{fig5-1}
 \includegraphics[width=\textwidth]{problem5_fig1.pdf}
 \caption{Apparent magnitude versus time for a Cepheid star. Actual measurements are represented as black points. Drawn in red is a piecewise cubic Hermite interpolation through the data.}
\end{figure} 

\begin{figure}[H]
 \label{fig5-2}
 \includegraphics[width=\textwidth]{problem5_fig2.pdf}
 \caption{Apparent magnitude versus time for a Cepheid star. Actual measurements are represented as black points. Drawn in red is a cubic spline interpolation through the data.}
\end{figure} 

\end{spacing}
\end{document}

%%%%%%%%%%%%%%%%%%%%%%%%%%%%%%%%%%%%%%%%%%%%%%%%%%%%%%%%%%%%%
